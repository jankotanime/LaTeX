\documentclass{article}

\usepackage{lipsum}
\usepackage{graphicx}
\usepackage[T1]{fontenc}

\title{Stepikowy Artykuł}
\author{Jan Gasztold}
\date{\today}

\begin{document}

\maketitle

\begin{abstract}
    Przykładowy artykuł losowe słowa generator
\end{abstract}

\section{Rozdział}

\subsection{Sekcja}

\underline{Każdy} chociaż \underline{raz} słyszał o tej firmie \textbf{produkującej zabawki}, ponieważ Lego opanowało rynek międzynarodowy już dziesiątki lat temu. Ale jak się im to udało? Głównie za sprawą świetnego pomysłu- zabawki te to nic innego jak \textbf{plastikowe części} - potocznie nazywane klockami - taki jeden klocek nie ma w sobie nic szczególnego, jednak specjalne wypustki i gniazda pozwalają na wiele kombinacji łączenia. Dzięki temu ograniczeniem dla osoby bawiącej się jest \textbf{tylko} jej wyobraźnia.

Aktualnie można nie tylko kupić skrzynki z pojedynczymi częściami, ale w większości przypadków ludzi interesują zestawy. Na początku zestawy były tylko spod przedsiębiorstwa Lego, jednak ogromnym sukcesem okazało się nawiązywanie współpracy \textit{Lego} z różnymi markami (najczęściej posiadające własne uniwersum filmowe) takimi jak ,,\textit{Star Wars}", ,,\textit{Harry Potter}", czy ,,\textit{Minecraft}". Dzięki temu zabawkami \textit{Lego} nie tylko fascynują się dzieci, ale także \underline{dorośli}, \underline{fani filmowi}, czy tzw. \textit{kolekcjonerzy}. \cite{autor1}

\subsection{Sekcja}

\lipsum[1]

\subsubsection{Podsekcja}

\lipsum[1]

\section{Rozdział}

\lipsum[1]

\subsection{Sekcja}

\subsubsection{Podsekcja}

\lipsum[1]

Tekst z zastosowanym \textbf{pogrubieniem}, \textit{kursywą} oraz \underline{podkreśleniem}.

\subsection{Matematyka}

\subsubsection{Wzór na równoważność masy i energii} 
$E=mc^2$.

\subsubsection{Wzór na sigme (suma ciągu)} 
$\sum_{i=1}^n=i$. \cite{autor2}

\section{Rysunki}

\begin{figure}[h]
    \centering
    \includegraphics[width=0.6\textwidth]{image2.png}
    \caption{Doktor Habilitowany Zdzisław Catowski}
    \label{fig:rys1}
\end{figure}

\lipsum[2]

\begin{figure}[h]
    \centering
    \begin{tabular}{|c|c|c|c|c|}
        \hline
        <2 & 2-3 & 3-4 & 4-5 & 5-6 \\
        \hline
        120-160 & 160-210 & 210-260 & 240-320 & 250-360 \\
        \hline
    \end{tabular}
    \caption{Tabela jedzenia kotów.}
    \label{tab:tabela1}
\end{figure}

\section{Odniesienia}

Odniesienie do Rysunku \ref{fig:rys1} oraz Tabeli \ref{tab:tabela1}. Jak można zauważyć Doktor Catowski był kotem

\centering
    \begin{minipage}{0.4\textwidth}
        \includegraphics[width=\linewidth]{image1.png}
        \label{fig:rys2}
    \end{minipage}
    \hspace{0.05\textwidth}
    \begin{minipage}{0.5\textwidth}
        Trygonometria
    \end{minipage}

\section{Podsumowanie}

\lipsum[2]

\begin{thebibliography}{9}
    \bibitem {autor1} Adamski, Maciej. \emph{Strusie i inne ssaki}. University of Gdansk, 2003.
    \bibitem{autor2} Catowski Zdzisław. \emph{Elementary Number Theory}. Kocie Wydawnictwo, 1998.
\end{thebibliography}

\end{document}
